\chapter{Tecnologias Aplicadas}

Com base no escopo que foi definido inicialmente para o aplicativo
Lixt optamos pela utilização de diferentes tecnologias que apresentam
vantagens durante o desenvolvimento.

Para o \textit{front-end} da aplicação escolhemos utilizar JavaScript
com a biblioteca React, pois a curva de apredizado dessa tecnologia é
curta e seu código é reutilizável, podendo ser inclusive utilizado em
aplicações mobile com o React Native (para Android e iOS).

Para o banco de dados foi definida a utilização do MySQL por possuir
uma grande variedade de funções e configurações que facilitam o
desenvolvimento.

Também no \textit{back-end}, vamos utilizar a linguagem Java
para o código-fonte, com o \textit{framework} Spring, em um
projeto Spring Boot. O Spring Boot pois traz mais
produtividade durante o desenvolvimento, permitindo concentrar os
esforços na implementação das regras de negócio do que com as
configurações de um projeto Web \cite{Alga2017}. Ainda nesse
sentido, vamos utilizar o Hibernate, que possibilita um
desenvolvimento mais àgil nas aplicações que possuem integração com
bancos de dados, como é o caso do aplicativo Lixt.

%%% Local Variables:
%%% mode: latex
%%% TeX-master: "../proposta"
%%% End:

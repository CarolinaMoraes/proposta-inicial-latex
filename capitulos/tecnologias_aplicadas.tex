\chapter{Tecnologias Aplicadas}

Com base no escopo que foi definido inicialmente para o aplicativo Lixt 
optamos pela utilização de diferentes tecnologias que apresentam vantagens 
durante o desenvolvimento.

Para o \textit{front-end} da aplicação escolhemos utilizar \textit{JavaScript} 
com a biblioteca \textit{React}, pois a curva de apredizado dessa tecnologia 
é curta e seu código é reutilizável, podendo ser inclusive utilizado em aplicações 
mobile com o \textit{React Native} (para \textit{Android} e \textit{iOS}).

Para o banco de dados foi definida a utilização do \textit{MySQL} por possuir uma 
grande variedade de funções e configurações que facilitam o desenvolvimento.

Também no \textit{back-end}, vamos utilizar a linguagem \textit{Java} para o código-fonte, com 
o \textit{framework} \textit{Spring}, em um projeto \textit{Spring Boot}. O 
\textit{Spring Boot} pois traz mais produtividade durante o desenvolvimento, permitindo concentrar os esforços na implementação das regras de negócio do que com as configurações de um projeto \textit{Web} \cite{Alga2017}. Ainda nesse sentido, vamos utilizar o \textit{Hibernate}, que possibilita um desenvolvimento mais àgil nas aplicações que possuem integração com bancos de dados, como é o caso do aplicativo Lixt.

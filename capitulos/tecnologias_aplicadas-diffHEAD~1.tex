\chapter{Tecnologias Aplicadas}

Com base no escopo que foi definido inicialmente para o aplicativo
Lixt optamos pela utilização de diferentes tecnologias que apresentam
vantagens durante o desenvolvimento.

Para o \textit{front-end} da aplicação escolhemos utilizar \DIFdelbegin \textit{\DIFdel{JavaScript}} 
%DIFAUXCMD
\DIFdelend \DIFaddbegin \DIFadd{JavaScript
}\DIFaddend com a biblioteca \DIFdelbegin \textit{\DIFdel{React}}%DIFAUXCMD
\DIFdelend \DIFaddbegin \DIFadd{React}\DIFaddend , pois a curva de apredizado dessa tecnologia é
curta e seu código é reutilizável, podendo ser inclusive utilizado em
aplicações mobile com o \DIFdelbegin \textit{\DIFdel{React Native}} %DIFAUXCMD
\DIFdel{(para }\textit{\DIFdel{Android}} %DIFAUXCMD
\DIFdel{e }\textit{\DIFdel{iOS}}%DIFAUXCMD
\DIFdelend \DIFaddbegin \DIFadd{React Native (para Android e iOS}\DIFaddend ).

Para o banco de dados foi definida a utilização do \DIFdelbegin \textit{\DIFdel{MySQL}} %DIFAUXCMD
\DIFdelend \DIFaddbegin \DIFadd{MySQL }\DIFaddend por possuir
uma grande variedade de funções e configurações que facilitam o
desenvolvimento.

Também no \textit{back-end}, vamos utilizar a linguagem \DIFdelbegin \textit{\DIFdel{Java}} %DIFAUXCMD
\DIFdelend \DIFaddbegin \DIFadd{Java
}\DIFaddend para o código-fonte, com o \textit{framework} \DIFdelbegin \textit{\DIFdel{Spring}}%DIFAUXCMD
\DIFdelend \DIFaddbegin \DIFadd{Spring}\DIFaddend , em um
projeto \DIFdelbegin \textit{\DIFdel{Spring Boot}}%DIFAUXCMD
\DIFdel{. O }\textit{\DIFdel{Spring Boot}} %DIFAUXCMD
\DIFdelend \DIFaddbegin \DIFadd{Spring Boot. O Spring Boot }\DIFaddend pois traz mais
produtividade durante o desenvolvimento, permitindo concentrar os
esforços na implementação das regras de negócio do que com as
configurações de um projeto \DIFdelbegin \textit{\DIFdel{Web}} %DIFAUXCMD
\DIFdelend \DIFaddbegin \DIFadd{Web }\DIFaddend \cite{Alga2017}. Ainda nesse
sentido, vamos utilizar o \DIFdelbegin \textit{\DIFdel{Hibernate}}%DIFAUXCMD
\DIFdelend \DIFaddbegin \DIFadd{Hibernate}\DIFaddend , que possibilita um
desenvolvimento mais àgil nas aplicações que possuem integração com
bancos de dados, como é o caso do aplicativo Lixt.
\DIFaddbegin 

%DIF > %% Local Variables:
%DIF > %% mode: latex
%DIF > %% TeX-master: "../proposta"
%DIF > %% End:
 \DIFaddend
\chapter{Introdução}

\section{Descrição do Problema}

Lorem ipsum dolor sia temet \cite[p. 3]{Kanan2015}.

\label{sec:objetivos}
\section{Objetivos}

O obetivo deste projeto é desenvolver um aplicativo multiplataforma,
focado no gerenciamento de listas de compras compartilhadas e que
apresente análises quanto à variação de preço para usuários
brasileiros. Para tal, é necessário que o aplicativo ofereça uma
interface intuitiva e responsiva, e que os sistemas back-end sejam
estáveis e rápidos.

Desta forma, uma arquitetura adequada seria uma aplicação em três
camadas, com um aplciativo front-end capaz de ser acessado tanto nos
dispositvos Android quanto iOS, e, eventualmente, web; que a api
disponível ao aplicativo seja bem estruturada conforme os padrões
REST; que os dados dos usúarios sejam armazenados de maneira segura e
criptografada; e que uma base dados volumosa seja fornececida de antemão para a
comodidade dos usuários quanto aos itens de mercado mais comuns no país.

\label{sec:solucao}
\section{Solução Proposta}

Tendo em vista o problema exposto e as limitações que serão levantadas
acerca dos concorrentes, propomos como solução o Aplicativo
\emph{Lixt}, que apresentará as seguintes funcionalidades, expostas em
ordem de dependências e nível de complexidade:

\begin{itemize}
\item Login\\
  O usuário deverá fazer login, de forma segura e privativa, para que
  possam acessar as funcionalidades que se seguem. Inicialmente, foi
  pensada uma solução na qual o login poderia ser opcional para as
  funcionalidades mais básicas, entretanto, isso impõe uma dificuldade
  na formulação de duas lógicas distintas no front-end.
\item Construir e gerenciar listas de compras\\
  O usário deverá ser capaz de criar e destruir listas de compras, adicionar ou
  remover itens de uma lista, especificar quantos itens devem ser
  comprados de cada produto e marcar itens como comprados. Esta
  funcionalidade é considerada a mais básica e essencial para a
  entrega do MVP.
\item Gerenciar categorias\\
  O usuário poderá atribuir categorias aos produtos adicionados,
  adicionar e deletar categorias, e filtrar itens de uma lista com base nelas.
\item Compartilhamento De Listas\\
  O usuário terá a possibilidade de compartilhar listas com outros
  usuários logados, e mudanças realizadas por um usário devem estar
  disponíveis para os outros usuários com acesso a lista.
\item Atribuição de itens para usuários em listas compartilhadas\\
  O usuário poderá atribuir, em listas compartilhadas, itens a pessoas
  com acesso a esta lista, como forma de direcionar quem deve comprar
  cada item.
\item Comentários em listas compartilhadas\\
  O usuário poderá adicionar, em um lista que tenha acesso,
  comentários associados a itens específicos com informações que
  considerar relevantes àquele produto, e estas notas devem estar
  disponíveis aos demais usuários que tenham acesso aquela lista
\item Gerenciamento de compras\\
  O usuário poderá realizar compras, vinculadas a uma lista e a um
  mercado, inserindo o preço pago por cada produto e, possivelmente,
  um desconto associado ao item. Quando o processo de for iniciado, o
  usuário verá o valor atualizado a ser pago no carrinho conforme os
  itens forem selecionados como ``pegos''.
\item Histórico de compras\\
  As compras realizadas deverão ser salvas e apresentadas conforme
  solicitadas pelo usuário, e apresentar o valor total em destaque, assim como a
  varição em relação as compras anteriores no mesmo mercado e com a
  mesma lista. Também, o preço de cada item deve apresentar a variação
  em relação ao valor anterior e posterior.
\item Análise estatística de compras\\
  Será apresentado ao usuário uma análise estatística dos itens
  comprados e dos valores pagos, em diversos níveis de
  especificidade. Planeja-se para a versão final do aplicativo que o
  usuário possa visulizar as seguintes variações:
  \begin{itemize}
  \item a variação de preço das diversas compras, levando em
    consideração ou não os mercados;
  \item quantas unidades de um item foi comprado por vez;
  \item a variação de preço do item a cada compra, levando os mercados
    em consideração;
  \item a variação de da média de preços de uma categoria,
    considerando ou não os mercados;
  \item a variação da média de preços de uma marca, considerando ou
    não os mercados.
  \end{itemize}
\end{itemize}

\section{Escopo do Projeto}
%%% Local Variables:
%%% mode: latex
%%% TeX-master: "../proposta"
%%% End:

\chapter{Gerenciamento do Projeto}
O gerenciamento do projeto se dará através da metodologia ágil Scrum,
a fim de gerar um desenvolvimento iterativo-incremental. As atividades
que compõem o projeto serão divididas em histórias e distribuídas em
sprints.  Como auxílio à organização de tarefas e fluxo contínuo de
desenvolvimento, a prática do Kanban será adotada.

Os membros da equipe exercerão os seguintes papéis:

\begin {itemize}
\item \textbf{Product Owner}: Fábio Mendes;
\item \textbf{Scrum Master}: Carolina de Moraes;
\item \textbf{Development Team}: Alkindar Rodrigues, Gabriely Bicigo,
  Leonardo Naoki e Mariana Zangrossi.
\end {itemize}

É importante ressaltar que, apesar dos papéis oficiais, todos os
membros ajudarão no desenvolvimento do projeto para que o mesmo possa
ser entregue dentro da data estimada.


Os softwares utilizados na gestão do projeto serão:

\begin {itemize}
\item \textbf{Trello} - Organização das tarefas;
\item \textbf{Google Meets} - Reuniões de alinhamento e demais
  cerimônias do Scrum;
\item \textbf{WhatsApp} - Comunicações instantâneas;
\item \textbf{Google Drive} - Armazenamento e compartilhamento dos
  arquivos, tais como seus backups.
\end {itemize}

